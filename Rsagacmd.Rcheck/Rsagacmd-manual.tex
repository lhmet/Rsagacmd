\nonstopmode{}
\documentclass[a4paper]{book}
\usepackage[times,inconsolata,hyper]{Rd}
\usepackage{makeidx}
\usepackage[utf8]{inputenc} % @SET ENCODING@
% \usepackage{graphicx} % @USE GRAPHICX@
\makeindex{}
\begin{document}
\chapter*{}
\begin{center}
{\textbf{\huge Package `Rsagacmd'}}
\par\bigskip{\large \today}
\end{center}
\inputencoding{utf8}
\ifthenelse{\boolean{Rd@use@hyper}}{\hypersetup{pdftitle = {Rsagacmd: Linking R with the Open-Source 'SAGA-GIS' Software}}}{}
\ifthenelse{\boolean{Rd@use@hyper}}{\hypersetup{pdfauthor = {Steven Pawley}}}{}
\begin{description}
\raggedright{}
\item[Type]\AsIs{Package}
\item[Title]\AsIs{Linking R with the Open-Source 'SAGA-GIS' Software}
\item[Version]\AsIs{0.4.2}
\item[Date]\AsIs{2023-10-15}
\item[Maintainer]\AsIs{Steven Pawley }\email{dr.stevenpawley@gmail.com}\AsIs{}
\item[Description]\AsIs{Provides an R scripting interface to the open-source 'SAGA-GIS' 
(System for Automated Geoscientific Analyses Geographical Information
System) software. 'Rsagacmd' dynamically generates R functions for every
'SAGA-GIS' geoprocessing tool based on the user's currently installed
'SAGA-GIS' version. These functions are contained within an S3 object
and are accessed as a named list of libraries and tools. This structure
facilitates an easier scripting experience by organizing the large number
of 'SAGA-GIS' geoprocessing tools (>700) by their respective library.
Interactive scripting can fully take advantage of code autocompletion tools
(e.g. in 'Rstudio'), allowing for each tools syntax to be quickly
recognized. Furthermore, the most common types of spatial data (via the
'terra', 'sp', and 'sf' packages) along with non-spatial data are
automatically passed from R to the 'SAGA-GIS' command line tool for
geoprocessing operations, and the results are loaded as the appropriate R
object. Outputs from individual 'SAGA-GIS' tools can also be chained using
pipes from the 'magrittr' and 'dplyr' packages to combine complex
geoprocessing operations together in a single statement. 'SAGA-GIS' is
available under a GPLv2 / LGPLv2 licence from
<}\url{https://sourceforge.net/projects/saga-gis/}\AsIs{> including Windows x86/x64
and macOS binaries. SAGA-GIS is also included in Debian/Ubuntu default software
repositories. Rsagacmd has currently been tested on 'SAGA-GIS' versions
from 2.3.1 to 9.2 on Windows, Linux and macOS.}
\item[License]\AsIs{GPL-3}
\item[Encoding]\AsIs{UTF-8}
\item[SystemRequirements]\AsIs{SAGA-GIS (>= 2.3.1)}
\item[RoxygenNote]\AsIs{7.2.3}
\item[Depends]\AsIs{R (>= 2.10)}
\item[Imports]\AsIs{generics, sf, terra (>= 1.7.0), stars, tools, utils, foreign,
stringr, rlang, tibble, processx, rvest}
\item[Suggests]\AsIs{dplyr, testthat (>= 3.0.0), covr}
\item[Config/testthat/edition]\AsIs{3}
\item[URL]\AsIs{}\url{https://stevenpawley.github.io/Rsagacmd/}\AsIs{}
\item[BugReports]\AsIs{}\url{https://github.com/stevenpawley/Rsagacmd/issues}\AsIs{}
\item[NeedsCompilation]\AsIs{no}
\item[Author]\AsIs{Steven Pawley [aut, cre]}
\end{description}
\Rdcontents{\R{} topics documented:}
\inputencoding{utf8}
\HeaderA{check\_output\_format}{Check the file extension of the output file to see if it is the same as the `raster\_format` or `vector\_format` settings. If a raster, such as a GeoTIFF is output directly from a SAGA-GIS tool but the raster format is set to SAGA, then this might work depending on the saga version but Rsagacmd will not know how to read the file.}{check.Rul.output.Rul.format}
\keyword{internal}{check\_output\_format}
%
\begin{Description}
Check the file extension of the output file to see if it is the same as the
`raster\_format` or `vector\_format` settings. If a raster, such as a GeoTIFF
is output directly from a SAGA-GIS tool but the raster format is set to SAGA,
then this might work depending on the saga version but Rsagacmd will not
know how to read the file.
\end{Description}
%
\begin{Usage}
\begin{verbatim}
check_output_format(x, raster_format, vector_format)
\end{verbatim}
\end{Usage}
%
\begin{Arguments}
\begin{ldescription}
\item[\code{x}] a `parameter` object that is an output parameter of a tool.

\item[\code{raster\_format}] the raster format.

\item[\code{vector\_format}] the vector format.
\end{ldescription}
\end{Arguments}
\inputencoding{utf8}
\HeaderA{convert\_sagaext\_r}{Ensure that the file extension for the SAGA raster format ends with .sdat for reading or writing SAGA grid objects in R.}{convert.Rul.sagaext.Rul.r}
\keyword{internal}{convert\_sagaext\_r}
%
\begin{Description}
This is used because the R raster/terra libraries expect to read and write
SAGA grid formats using the '.sdat' file extension, not '.sgrd'.
\end{Description}
%
\begin{Usage}
\begin{verbatim}
convert_sagaext_r(fp)
\end{verbatim}
\end{Usage}
%
\begin{Arguments}
\begin{ldescription}
\item[\code{fp}] file path to raster writing
\end{ldescription}
\end{Arguments}
%
\begin{Value}
a character vector with the corrected file extensions to read SAGA
sgrd files back into R.
\end{Value}
\inputencoding{utf8}
\HeaderA{create\_alias}{Generates a syntactically-correct R name based on a SAGA-GIS identifier}{create.Rul.alias}
\keyword{internal}{create\_alias}
%
\begin{Description}
SAGA-GIS identifiers sometimes cannot represent syntactically-correct names
in R because they start with numbers or have spaces. They are also all in
uppercase which is ugly to refer to in code. This function creates an
alternative/alias identifier.
\end{Description}
%
\begin{Usage}
\begin{verbatim}
create_alias(identifier)
\end{verbatim}
\end{Usage}
%
\begin{Arguments}
\begin{ldescription}
\item[\code{identifier}] A character with the identifier.
\end{ldescription}
\end{Arguments}
%
\begin{Value}
A character with a syntactically-correct alias.
\end{Value}
\inputencoding{utf8}
\HeaderA{create\_function}{Function generate text that will be parsed into R code}{create.Rul.function}
\keyword{internal}{create\_function}
%
\begin{Description}
Internal variable `args` is derived by capturing the names and values of the
calling function. The interval `senv` variable is the SAGA-GIS library
settings (generated by `saga\_env`) and comes from the environment of when the
dynamic function was generated.
\end{Description}
%
\begin{Usage}
\begin{verbatim}
create_function(lib, tool)
\end{verbatim}
\end{Usage}
%
\begin{Arguments}
\begin{ldescription}
\item[\code{lib}] A character, name of SAGA-GIS library.

\item[\code{tool}] A character, name of SAGA-GIS tool.
\end{ldescription}
\end{Arguments}
%
\begin{Value}
A character, text that is to be parsed into a function definition.
\end{Value}
\inputencoding{utf8}
\HeaderA{create\_tool}{Generates list of options for a SAGA-GIS tool}{create.Rul.tool}
\keyword{internal}{create\_tool}
%
\begin{Description}
Parses the html table for a SAGA-GIS tool into a list of identifiers,
options, defaults and constraints
\end{Description}
%
\begin{Usage}
\begin{verbatim}
create_tool(tool_information, tool_options, description, html_file)
\end{verbatim}
\end{Usage}
%
\begin{Arguments}
\begin{ldescription}
\item[\code{tool\_information}] list

\item[\code{tool\_options}] list

\item[\code{description}] the description text for the tool that has been scraped
from the help documentation

\item[\code{html\_file}] the name of the html file for the tool's documentation.
Stored to help linking with online documentation.
\end{ldescription}
\end{Arguments}
%
\begin{Value}
A `saga\_tool` object containing:
+ `tool\_name` A syntactically-correct name for the tool.
+ `description` The tool's description.
+ `author` The tool's author.
+ `tool\_cmd` The command to use for saga\_cmd to execute tool.
+ `tool\_id` The tool's ID.
+ `parameters` A named list of the tool's parameter objects.
+ `html\_file` The html document name.
\end{Value}
\inputencoding{utf8}
\HeaderA{create\_tool\_overrides}{Apply manually-defined changes to specific tools}{create.Rul.tool.Rul.overrides}
\keyword{internal}{create\_tool\_overrides}
%
\begin{Description}
Used to manually alter or add parameters for specific tools outside of what
has been defined based on the output of saga\_cmd --create-docs
\end{Description}
%
\begin{Usage}
\begin{verbatim}
create_tool_overrides(tool_name, params)
\end{verbatim}
\end{Usage}
%
\begin{Arguments}
\begin{ldescription}
\item[\code{tool\_name}] character, name of the tool. This is the alias name used by
Rsagacmd, i.e. the tool name without spaces, all lowercase etc.

\item[\code{params}] the `parameters` object for the tool
\end{ldescription}
\end{Arguments}
%
\begin{Value}
the altered `parameters` object
\end{Value}
\inputencoding{utf8}
\HeaderA{drop\_parameters}{Drops unused/empty parameters from a `parameters` object}{drop.Rul.parameters}
\keyword{internal}{drop\_parameters}
%
\begin{Description}
Drops unused/empty parameters from a `parameters` object
\end{Description}
%
\begin{Usage}
\begin{verbatim}
drop_parameters(params)
\end{verbatim}
\end{Usage}
%
\begin{Arguments}
\begin{ldescription}
\item[\code{params}] A `parameters` object
\end{ldescription}
\end{Arguments}
%
\begin{Value}
A `parameters` object with empty `parameter` objects removed
\end{Value}
\inputencoding{utf8}
\HeaderA{extract\_tool}{Internal function to extract information from a `saga\_tool` object}{extract.Rul.tool}
\keyword{internal}{extract\_tool}
%
\begin{Description}
Internal function to extract information from a `saga\_tool` object
\end{Description}
%
\begin{Usage}
\begin{verbatim}
extract_tool(x)
\end{verbatim}
\end{Usage}
%
\begin{Arguments}
\begin{ldescription}
\item[\code{x}] a `saga\_tool` object
\end{ldescription}
\end{Arguments}
%
\begin{Value}
the intervals of a `saga\_tool`
\end{Value}
\inputencoding{utf8}
\HeaderA{mrvbf\_threshold}{Calculate the t\_slope value based on DEM resolution for MRVBF}{mrvbf.Rul.threshold}
%
\begin{Description}
Calculates the t\_slope value for the Multiresolution Index of Valley Bottom
Flatness (Gallant and Dowling, 2003) based on input DEM resolution. MRVBF
identified valley bottoms based on classifying slope angle and identifying
low areas by ranking elevation in respect to the surrounding topography
across a range of DEM resolutions. The MRVBF algorithm was developed using a
25 m DEM, and so if the input DEM has a different resolution then the slope
threshold t\_slope needs to be adjusted from its default value of 16 in order
to maintain the relationship between slope and DEM resolution. This function
provides a convenient way to perform that calculation.
\end{Description}
%
\begin{Usage}
\begin{verbatim}
mrvbf_threshold(res)
\end{verbatim}
\end{Usage}
%
\begin{Arguments}
\begin{ldescription}
\item[\code{res}] numeric, DEM resolution
\end{ldescription}
\end{Arguments}
%
\begin{Value}
numeric, t\_slope value for MRVBF
\end{Value}
%
\begin{Examples}
\begin{ExampleCode}
mrvbf_threshold(res = 10)
\end{ExampleCode}
\end{Examples}
\inputencoding{utf8}
\HeaderA{parameter}{Parameter class}{parameter}
\keyword{internal}{parameter}
%
\begin{Description}
Stores metadata associated with each SAGA-GIS tool parameter.
\end{Description}
%
\begin{Usage}
\begin{verbatim}
parameter(type, name, alias, identifier, description, constraints)
\end{verbatim}
\end{Usage}
%
\begin{Arguments}
\begin{ldescription}
\item[\code{type}] A character to describe the data type of the parameter. One of
"input", "output", "Grid", "Grid list", "Shapes", "Shapes list", "Table",
"Static table", "Table list", "File path", "field", "Integer", "Choice",
"Floating point", "Boolean", "Long text", "Text.

\item[\code{name}] A character with the long name of the parameter.

\item[\code{alias}] A syntactically correct alias for the identifier.

\item[\code{identifier}] A character with the identifier of the parameter used by
saga\_cmd.

\item[\code{description}] A character with the description of the parameter.

\item[\code{constraints}] A character describing the parameters constraints.
\end{ldescription}
\end{Arguments}
%
\begin{Value}
A `parameter` class object.
\end{Value}
\inputencoding{utf8}
\HeaderA{parameters}{Generates a list of `parameter` objects for a SAGA-GIS tool}{parameters}
\keyword{internal}{parameters}
%
\begin{Description}
Each `parameter` object contains information about the datatype, permissible
values and input/output settings associated with each identifier for a
SAGA-GIS tool.
\end{Description}
%
\begin{Usage}
\begin{verbatim}
parameters(tool_options)
\end{verbatim}
\end{Usage}
%
\begin{Arguments}
\begin{ldescription}
\item[\code{tool\_options}] A data.frame containing the table that refers to the
SAGA-GIS tool parameter options.
\end{ldescription}
\end{Arguments}
%
\begin{Value}
A `parameters` object
\end{Value}
\inputencoding{utf8}
\HeaderA{parse\_options}{Convenience function to join together the saga\_cmd option:value pairs}{parse.Rul.options}
\keyword{internal}{parse\_options}
%
\begin{Description}
Convenience function to join together the saga\_cmd option:value pairs
\end{Description}
%
\begin{Usage}
\begin{verbatim}
parse_options(key, value)
\end{verbatim}
\end{Usage}
%
\begin{Arguments}
\begin{ldescription}
\item[\code{key}] character, the saga\_cmd option such as "DEM".

\item[\code{value}] character, the value of the option.
\end{ldescription}
\end{Arguments}
%
\begin{Value}
character, a joined option:value pair such as "-DEM:mygrid.tif"
\end{Value}
\inputencoding{utf8}
\HeaderA{print.saga\_tool}{Generic function to display help and usage information for any SAGA-GIS tool}{print.saga.Rul.tool}
%
\begin{Description}
Displays a tibble containing the name of the tool's parameters, the argument
name used by Rsagacmd, the identifier used by the SAGA-GIS command line, and
additional descriptions, default and options/constraints.
\end{Description}
%
\begin{Usage}
\begin{verbatim}
## S3 method for class 'saga_tool'
print(x, ...)
\end{verbatim}
\end{Usage}
%
\begin{Arguments}
\begin{ldescription}
\item[\code{x}] A `saga\_tool` object.

\item[\code{...}] Additional arguments to pass to print. Currently not used.
\end{ldescription}
\end{Arguments}
%
\begin{Examples}
\begin{ExampleCode}
## Not run: 
# Initialize a saga object
saga <- saga_gis()

# Display usage information on a tool
print(saga$ta_morphometry$slope_aspect_curvature)

# Or simply:
saga$ta_morphometry$slope_aspect_curvature

## End(Not run)
\end{ExampleCode}
\end{Examples}
\inputencoding{utf8}
\HeaderA{read\_grid}{Read a raster data set that is output by saga\_cmd}{read.Rul.grid}
\keyword{internal}{read\_grid}
%
\begin{Description}
Read a raster data set that is output by saga\_cmd
\end{Description}
%
\begin{Usage}
\begin{verbatim}
read_grid(x, backend)
\end{verbatim}
\end{Usage}
%
\begin{Arguments}
\begin{ldescription}
\item[\code{x}] list, a `options` object that was created by the `create\_tool`
function that contains the parameters for a particular tool and its
outputs.

\item[\code{backend}] character, either "raster", "terra" or "stars".
\end{ldescription}
\end{Arguments}
%
\begin{Value}
either a `raster` or `SpatRaster` object
\end{Value}
\inputencoding{utf8}
\HeaderA{read\_grid\_list}{Read a semi-colon separated list of grids that are output by saga\_cmd}{read.Rul.grid.Rul.list}
\keyword{internal}{read\_grid\_list}
%
\begin{Description}
Read a semi-colon separated list of grids that are output by saga\_cmd
\end{Description}
%
\begin{Usage}
\begin{verbatim}
read_grid_list(x, backend)
\end{verbatim}
\end{Usage}
%
\begin{Arguments}
\begin{ldescription}
\item[\code{x}] list, a `options` object that was created by the `create\_tool`
function that contains the parameters for a particular tool and its
outputs.

\item[\code{backend}] character, either "raster" or "terra"
\end{ldescription}
\end{Arguments}
%
\begin{Value}
list, containing multiple `raster` or `SpatRaster` objects.
\end{Value}
\inputencoding{utf8}
\HeaderA{read\_output}{Primary function to read data sets (raster, vector, tabular) that are output by saga\_cmd}{read.Rul.output}
\keyword{internal}{read\_output}
%
\begin{Description}
Primary function to read data sets (raster, vector, tabular) that are output
by saga\_cmd
\end{Description}
%
\begin{Usage}
\begin{verbatim}
read_output(output, raster_backend, vector_backend, .intern, .all_outputs)
\end{verbatim}
\end{Usage}
%
\begin{Arguments}
\begin{ldescription}
\item[\code{output}] list, a `options` object that was created by the `create\_tool`
function that contains the parameters for a particular tool and its
outputs.

\item[\code{raster\_backend}] character, either "raster" or "terra"

\item[\code{vector\_backend}] character, either "sf", "SpatVector" or
"SpatVectorProxy"

\item[\code{.intern}] logical, whether to load the output as an R object
\end{ldescription}
\end{Arguments}
%
\begin{Value}
the loaded objects, or NULL is `.intern = FALSE`.
\end{Value}
\inputencoding{utf8}
\HeaderA{read\_shapes}{Read a spatial vector data set that is output by saga\_cmd}{read.Rul.shapes}
\keyword{internal}{read\_shapes}
%
\begin{Description}
Read a spatial vector data set that is output by saga\_cmd
\end{Description}
%
\begin{Usage}
\begin{verbatim}
read_shapes(x, vector_backend)
\end{verbatim}
\end{Usage}
%
\begin{Arguments}
\begin{ldescription}
\item[\code{x}] list, a `options` object that was created by the `create\_tool`
function that contains the parameters for a particular tool and its
outputs.

\item[\code{vector\_backend}] character for vector backend to use.
\end{ldescription}
\end{Arguments}
%
\begin{Value}
an `sf` object.
\end{Value}
\inputencoding{utf8}
\HeaderA{read\_srtm}{Get path to the example DEM data}{read.Rul.srtm}
%
\begin{Description}
Rsagacmd comes bundled with a small tile of example Digital Elevation Model
(DEM) data from the NASA Shuttle Radar Topography Mission Global 1 arc second
V003. This data is stored in GeoTIFF format in `inst/extdata`.
\end{Description}
%
\begin{Usage}
\begin{verbatim}
read_srtm()
\end{verbatim}
\end{Usage}
%
\begin{Details}
The dataset contains the land surface elevation of an area located near
Jasper, Alberta, Canada, with the coordinate reference system (CRS) EPSG code
of 3402 (NAD83(CSRS) / Alberta 10-TM (Forest)).

To access the data, use the convenience function of `read\_srtm()` to load
the data as a `terra::SpatRaster` object.
\end{Details}
%
\begin{Examples}
\begin{ExampleCode}
library(Rsagacmd)
library(terra)

dem <- read_srtm()
plot(dem)
\end{ExampleCode}
\end{Examples}
\inputencoding{utf8}
\HeaderA{read\_table}{Read a tabular data set that is output by saga\_cmd}{read.Rul.table}
\keyword{internal}{read\_table}
%
\begin{Description}
Read a tabular data set that is output by saga\_cmd
\end{Description}
%
\begin{Usage}
\begin{verbatim}
read_table(x)
\end{verbatim}
\end{Usage}
%
\begin{Arguments}
\begin{ldescription}
\item[\code{x}] list, a `options` object that was created by the `create\_tool`
function that contains the parameters for a particular tool and its
outputs.
\end{ldescription}
\end{Arguments}
%
\begin{Value}
a `tibble`.
\end{Value}
\inputencoding{utf8}
\HeaderA{reexports}{Objects exported from other packages}{reexports}
\aliasA{tidy}{reexports}{tidy}
\keyword{internal}{reexports}
%
\begin{Description}
These objects are imported from other packages. Follow the links
below to see their documentation.

\begin{description}

\item[generics] \code{\LinkA{tidy}{tidy}}

\end{description}
\end{Description}
\inputencoding{utf8}
\HeaderA{Rsagacmd}{Rsagacmd: Linking R with the open-source SAGA-GIS software.}{Rsagacmd}
\aliasA{Rsagacmd-package}{Rsagacmd}{Rsagacmd.Rdash.package}
\keyword{internal}{Rsagacmd}
%
\begin{Description}
\pkg{Rsagacmd} provides an R scripting interface to the open-source System
for Automated Geoscientific Analyses Geographical Information System software
\Rhref{https://sourceforge.net/projects/saga-gis/}{SAGA-GIS}. The current
version has been tested using SAGA-GIS versions 2.3.1 to 9.2 on Windows
(x64), macOS and Linux. Rsagacmd provides a functional approach to scripting
with SAGA-GIS by dynamically generating R functions for every SAGA-GIS tool
based on the user's current SAGA-GIS installation. These functions are
generated by the \code{\LinkA{saga\_gis}{saga.Rul.gis}} function and are included within an
S3 object as a named list of libraries and tools. This structure facilitates
an easier scripting experience by organizing the large number of SAGA-GIS
tools (>700) by their respective library. Interactive scripting can also
fully take advantage of code autocompletion tools (e.g. in
\Rhref{https://posit.co}{Rstudio}), allowing for each tool's syntax to
be quickly recognized. Furthermore, the most common types of spatial data
(rasters using the \pkg{terra} and \pkg{stars} packages, and vector data
using \pkg{sp}, \pkg{sf} or \pkg{terra} packages) along with non-spatial data
are seamlessly passed from R to the SAGA-GIS command line tool for
geoprocessing operations, and the results are automatically loaded as the
appropriate R object. Outputs from individual SAGA-GIS tools can also be
chained using pipes from the \pkg{magrittr} and \pkg{dplyr} packages to chain
complex geoprocessing operations together in a single statement.
\end{Description}
%
\begin{Section}{Handling of geospatial and tabular data}
Rsagacmd aims to facilitate
a seamless interface to the open-source SAGA-GIS by providing access to all
SAGA-GIS geoprocessing tools in a 'R-like' manner. In addition to
generating R functions that correspond to each SAGA-GIS tool, Rsagacmd
automatically handles the passing of geospatial and tabular data contained
from the R environment to SAGA-GIS.

Rsagacmd uses the SAGA-GIS command line interface to perform geoprocessing
operations. Therefore, spatial data can be passed to any Rsagacmd function as
a path to the input data, assuming that the data is stored in the appropriate
file formats (e.g. GDAL-supported single-band rasters, OGR supported vector
data, and comma- or tab-delimited text files for tabular data). In addition,
Rsagacmd also supports the following R object classes to pass data to
SAGA-GIS, and to load the results back into the R environment:
\begin{itemize}

\item{} Raster data handling is provided by the R \pkg{terra} and \pkg{stars}
packages. Raster-based outputs from SAGA-GIS tools are loaded as SpatRaster
or stars objects. For more details, see the 'Handling of raster data'.
\item{} Vector features that result from SAGA-GIS geoprocessing operations are
output in ESRI Shapefile format and are loaded into the R environment as
simple features objects
\item{} Tabular data from SAGA-GIS tools are loaded as data frames

\end{itemize}

The results from tools that return multiple outputs are loaded into the R
environment as a named list of the appropriate R object classes.
\end{Section}
%
\begin{Section}{Multi-band raster data}
SAGA-GIS does not handle multi-band rasters
and the native SAGA GIS Binary file format (.sgrd) supports only single
band data. Therefore when passing raster data to most SAGA-GIS tools using
Rsagacmd, the data should represent single raster bands. Subsetting of
raster data is performed automatically by Rsagacmd in the case of when a
single band from a multiband SpatRaster or stars object is passed to a
SAGA-GIS tool. This occurs in by either passing the filename of the raster
to the SAGA-GIS command line, or by writing the data to a temporary file.
\end{Section}
%
\begin{Section}{Combining SAGA-GIS tools with pipes}
For convenience, outputs from
SAGA-GIS tools are automatically saved to tempfiles if outputs are not
explicitly stated as arguments when calling the function. This was
implemented so that the user can create complex workflows based on little
code. It is also means that several processing steps can be combined or
chained in a convenient manner using pipes from the \pkg{magritrr} package.
When using pipes, all of the intermediate processing steps are dealt with
automatically by saving the outputs as tempfiles, and then in turn passing
the output to the next function in the chain. Note that when dealing with
high-resolution and/or larger raster data, these tempfiles can start to
consume a significant amount of disk space during a session. If required,
these temporary files can be cleaned during the session using the
saga\_remove\_tmpfiles function.
\end{Section}
%
\begin{Author}
\strong{Maintainer}: Steven Pawley \email{dr.stevenpawley@gmail.com}

\end{Author}
%
\begin{SeeAlso}
Useful links:
\begin{itemize}

\item{} \url{https://stevenpawley.github.io/Rsagacmd/}
\item{} Report bugs at \url{https://github.com/stevenpawley/Rsagacmd/issues}

\end{itemize}


\end{SeeAlso}
%
\begin{Examples}
\begin{ExampleCode}
## Not run: 
library(Rsagacmd)
library(magrittr)

# initialize a saga object
saga <- saga_gis(opt_lib = c("grid_calculus", "ta_morphometry"))

# example of executing a tool using a tempfile to store the tool outputs
dem <- saga$grid_calculus$random_terrain(radius = 100, iterations = 500)

# Example of chaining operations using pipes and using tempfile to
# store tool outputs
tri <- dem %>%
  saga$ta_morphometry$terrain_ruggedness_index_tri()

# Remove tempfiles generated by Rsagacmd during a session
saga_remove_tmpfiles(h = 0)

## End(Not run)
\end{ExampleCode}
\end{Examples}
\inputencoding{utf8}
\HeaderA{run\_cmd}{Prepares the statement and runs the external saga\_cmd executable}{run.Rul.cmd}
\keyword{internal}{run\_cmd}
%
\begin{Description}
Prepares the statement and runs the external saga\_cmd executable
\end{Description}
%
\begin{Usage}
\begin{verbatim}
run_cmd(saga_cmd, saga_config, lib, tool_cmd, args, verbose)
\end{verbatim}
\end{Usage}
%
\begin{Arguments}
\begin{ldescription}
\item[\code{saga\_cmd}] character, name of the saga\_cmd executable or alias.

\item[\code{saga\_config}] character, path to the saga configuration "ini" file.

\item[\code{lib}] character, name of the selected library.

\item[\code{tool\_cmd}] character, name of the selected tool.

\item[\code{args}] named list of tool options, such as list(DEM = "mygrid.tif",
RADIUS = 3).

\item[\code{verbose}] logical, whether to show all saga\_cmd messages on the R
console.
\end{ldescription}
\end{Arguments}
%
\begin{Value}
list, output from `processx::run()`
\end{Value}
\inputencoding{utf8}
\HeaderA{saga\_configure}{Generates a custom saga\_cmd configuration file}{saga.Rul.configure}
%
\begin{Description}
Creates and edits a saga\_cmd configuration file in order to change saga\_cmd
settings related to file caching and number of available processor cores.
Intended to be used internally by \code{\LinkA{saga\_gis}{saga.Rul.gis}}
\end{Description}
%
\begin{Usage}
\begin{verbatim}
saga_configure(
  senv,
  grid_caching = FALSE,
  grid_cache_threshold = 100,
  grid_cache_dir = NULL,
  cores = NULL,
  saga_vers
)
\end{verbatim}
\end{Usage}
%
\begin{Arguments}
\begin{ldescription}
\item[\code{senv}] A saga environment object. Contains the SAGA-GIS environment and
settings.

\item[\code{grid\_caching}] Whether to use file caching. The default is FALSE.

\item[\code{grid\_cache\_threshold}] Any number to use as a threshold (in Mb) before
file caching for loaded raster data is activated.

\item[\code{grid\_cache\_dir}] Optionally specify a path to the used directory for
temporary files. The default uses `base::tempdir`.

\item[\code{cores}] An integer specifying the maximum number of processing cores.
Needs to be set to 1 if file caching is activated because file caching in
SAGA-GIS is not thread-safe.

\item[\code{saga\_vers}] A `numeric\_version` that specifies the version of SAGA-GIS.
The generation of a saga\_cmd configuration file is only valid for versions
> 4.0.0.
\end{ldescription}
\end{Arguments}
%
\begin{Value}
A character that specifies the path to custom saga\_cmd initiation
file.
\end{Value}
\inputencoding{utf8}
\HeaderA{saga\_docs}{Browse the online documentation for a saga\_tool}{saga.Rul.docs}
%
\begin{Description}
Browse the online documentation for a saga\_tool
\end{Description}
%
\begin{Usage}
\begin{verbatim}
saga_docs(saga_tool)
\end{verbatim}
\end{Usage}
%
\begin{Arguments}
\begin{ldescription}
\item[\code{saga\_tool}] a saga\_tool object
\end{ldescription}
\end{Arguments}
%
\begin{Examples}
\begin{ExampleCode}
## Not run: 
library(Rsagacmd)

saga <- saga_gis()

saga_docs(saga$ta_morphometry$slope_aspect_curvature)

## End(Not run)
\end{ExampleCode}
\end{Examples}
\inputencoding{utf8}
\HeaderA{saga\_env}{Parses valid SAGA-GIS libraries and tools into a nested list of functions}{saga.Rul.env}
%
\begin{Description}
Establishes the link to SAGA GIS by generating a SAGA help file and parsing
all libraries, tools and options from the help files into a nested list of
library, module and options, that are contained within an saga environment
object object. Intended to be used internally by \code{\LinkA{saga\_gis}{saga.Rul.gis}}
\end{Description}
%
\begin{Usage}
\begin{verbatim}
saga_env(
  saga_bin = NULL,
  opt_lib = NULL,
  raster_backend = "terra",
  vector_backend = "sf"
)
\end{verbatim}
\end{Usage}
%
\begin{Arguments}
\begin{ldescription}
\item[\code{saga\_bin}] An optional character vector to specify the path to the
saga\_cmd executable. Otherwise the function will perform a search for
saga\_cmd.

\item[\code{opt\_lib}] A character vector of a subset of SAGA-GIS tool libraries to
generate dynamic functions that map to each tool. Used to save time if you
only want to import a single library.

\item[\code{raster\_backend}] A character vector to specify the library to use for
handling raster data. Currently, either "terra" or "stars" is
supported. The default is "terra".

\item[\code{vector\_backend}] A character to specify the library to use for handling
vector data. Currently, either "sf", "SpatVector" or "SpatVectorProxy" is
supported. The default is "sf".
\end{ldescription}
\end{Arguments}
%
\begin{Value}
A saga environment S3 object containing paths, settings and a nested
list of libraries tools and options.
\end{Value}
\inputencoding{utf8}
\HeaderA{saga\_execute}{Function to execute SAGA-GIS commands through the command line tool}{saga.Rul.execute}
%
\begin{Description}
Intended to be used internally by each function
\end{Description}
%
\begin{Usage}
\begin{verbatim}
saga_execute(
  lib,
  tool,
  senv,
  .intern = NULL,
  .all_outputs = NULL,
  .verbose = NULL,
  ...
)
\end{verbatim}
\end{Usage}
%
\begin{Arguments}
\begin{ldescription}
\item[\code{lib}] A character specifying the name of SAGA-GIS library to execute.

\item[\code{tool}] A character specifying the name of SAGA-GIS tool to execute.

\item[\code{senv}] A saga environment object.

\item[\code{.intern}] A logical specifying whether to load the outputs from the
SAGA-GIS geoprocessing operation as an R object.

\item[\code{.all\_outputs}] A logical to specify whether to automatically output all
results from the selected SAGA tool and load them results as R objects
(default = TRUE). If .all\_outputs = FALSE then the file paths to store the
tool's results will have to be manually specified in the arguments.

\item[\code{.verbose}] Option to output all message during the execution of
saga\_cmd. Overrides the saga environment setting.

\item[\code{...}] Named arguments and values for SAGA tool.
\end{ldescription}
\end{Arguments}
%
\begin{Value}
output of SAGA-GIS tool loaded as an R object.
\end{Value}
\inputencoding{utf8}
\HeaderA{saga\_gis}{Initiate a SAGA-GIS geoprocessor object}{saga.Rul.gis}
%
\begin{Description}
Dynamically generates functions to all valid SAGA-GIS libraries and tools.
These functions are stored within a saga S3 object as a named list of
functions
\end{Description}
%
\begin{Usage}
\begin{verbatim}
saga_gis(
  saga_bin = NULL,
  grid_caching = FALSE,
  grid_cache_threshold = 100,
  grid_cache_dir = NULL,
  cores = NULL,
  raster_backend = "terra",
  vector_backend = "sf",
  raster_format = "SAGA",
  vector_format = c("ESRI Shapefile", "GeoPackage"),
  all_outputs = TRUE,
  intern = TRUE,
  opt_lib = NULL,
  temp_path = NULL,
  verbose = FALSE
)
\end{verbatim}
\end{Usage}
%
\begin{Arguments}
\begin{ldescription}
\item[\code{saga\_bin}] The path to saga\_cmd executable. If this argument
is not supplied then an automatic search for the saga\_cmd executable will
be performed.

\item[\code{grid\_caching}] A logical whether to use file caching in saga\_cmd
geoprocessing operations for rasters that are too large to fit into memory.

\item[\code{grid\_cache\_threshold}] A number to act as a threshold (in Mb) before
file caching is activated for loaded raster data.

\item[\code{grid\_cache\_dir}] The path to directory for temporary files generated by
file caching. If not provided then the result from `base::tempdir()` is
used.

\item[\code{cores}] An integer for the maximum number of processing cores. By
default all cores are utilized. Needs to be set to 1 if file caching is
activated.

\item[\code{raster\_backend}] A character vector to specify the library to use for
handling raster data. Supported options are "terra" or "stars".
The default is "terra".

\item[\code{vector\_backend}] A character to specify the library to use for handling
vector data. Currently, "sf", "SpatVector" or "SpatVectorProxy" is
supported. The default is "sf", however for large vector datasets, using
the "SpatVectorProxy" backend from the `terra` package has performance
advantages because it allows file-based which can reduce repeated
reading/writing when passing data between R and SAGA-GIS.

\item[\code{raster\_format}] A character to specify the default format used to save
raster data sets that are produced by SAGA-GIS. Available options are one
of "SAGA", "SAGA Compressed" or "GeoTIFF". The default is "SAGA".

\item[\code{vector\_format}] A character to specify the default format used for
vector data sets that are produced by SAGA-GIS, and also used to save
in-memory objects to be read by SAGA-GIS. Available options are of of "ESRI
Shapefile", "GeoPackage", or "GeoJSON". The default is "ESRI Shapefile" for
SAGA versions < 7.0 and GeoPackage for more recent versions. Attempting to
use anything other than "ESRI Shapefile" for SAGA-GIS versions < 7.0 will
raise an error.

\item[\code{all\_outputs}] A logical to indicate whether to automatically use
temporary files to store all output data sets from each SAGA-GIS tool.
Default = TRUE. This argument can be overridden by the `.all\_outputs`
parameter on each individual SAGA-GIS tool function that is generated by
`Rsagacmd::saga\_gis()`.

\item[\code{intern}] A logical to indicate whether to load the SAGA-GIS
geoprocessing results as an R object, default = TRUE. For instance, if a
raster grid is output by SAGA-GIS then this will be loaded as either as
a `SpatRaster` or `stars` object, depending on the `raster\_backend`
setting that is used. Vector data sets are always loaded as `sf` objects,
and tabular data sets are loaded as tibbles. The `intern` settings for the
`saga` object can be overridden for individual tools using the `.intern`
argument.

\item[\code{opt\_lib}] A character vector with the names of a subset of SAGA-GIS
libraries. Used to link only a subset of named SAGA-GIS tool libraries,
rather than creating functions for all available tool libraries.

\item[\code{temp\_path}] The path to use to store any temporary files that are
generated as data is passed between R and SAGA-GIS. If not specified, then
the system `base::tempdir()` is used.

\item[\code{verbose}] Logical to indicate whether to output all messages made during
SAGA-GIS commands to the R console. Default = FALSE. This argument can be
overriden by using the `.verbose` argument on each individual SAGA-GIS tool
function that is generated by `Rsagacmd::saga\_gis()`.
\end{ldescription}
\end{Arguments}
%
\begin{Value}
A S3 `saga` object containing a nested list of functions for SAGA-GIS
libraries and tools.
\end{Value}
%
\begin{Examples}
\begin{ExampleCode}
## Not run: 
# Initialize a saga object
library(Rsagacmd)
library(terra)

saga <- saga_gis()

# Alternatively initialize a saga object using file caching to handle large
# raster files
saga <- saga_gis(grid_caching = TRUE, grid_cache_threshold = 250, cores = 1)

# Example terrain analysis
# Generate a random DEM
dem <- saga$grid_calculus$random_terrain(radius = 100)

# Use Rsagacmd to calculate the Terrain Ruggedness Index
tri <- saga$ta_morphometry$terrain_ruggedness_index_tri(dem = dem)
plot(tri)

# Optionally run command and do not load result as an R object
saga$ta_morphometry$terrain_ruggedness_index_tri(dem = dem, .intern = FALSE)

# Initialize a saga object but do not automatically save all results to
# temporary files to load into R. Use this if you are explicitly saving each
# output because this will save disk space by not saving results from tools
# that output multiple results that you may be want to keep.
saga <- saga_gis(all_outputs = FALSE)

## End(Not run)
\end{ExampleCode}
\end{Examples}
\inputencoding{utf8}
\HeaderA{saga\_remove\_tmpfiles}{Removes temporary files created by Rsagacmd}{saga.Rul.remove.Rul.tmpfiles}
%
\begin{Description}
For convenience, functions in the Rsagacmd package create temporary files if
any outputs for a SAGA-GIS tool are not specified as arguments. Temporary
files in R are automatically removed at the end of each session. However,
when dealing with raster data, these temporary files potentially can consume
large amounts of disk space. These temporary files can be observed during a
session by using the saga\_show\_tmpfiles function, and can be removed using
the saga\_remove\_tmpfiles function. Note that this function also removes any
accompanying files, i.e. the '.prj' and '.shx' files that may be written as
part of writing a ESRI Shapefile '.shp' format
\end{Description}
%
\begin{Usage}
\begin{verbatim}
saga_remove_tmpfiles(h = 0)
\end{verbatim}
\end{Usage}
%
\begin{Arguments}
\begin{ldescription}
\item[\code{h}] Remove temporary files that are older than h (in number of hours).
\end{ldescription}
\end{Arguments}
%
\begin{Value}
Nothing is returned.
\end{Value}
%
\begin{Examples}
\begin{ExampleCode}
## Not run: 
# Remove all temporary files generated by Rsagacmd
saga_remove_tmpfiles(h = 0)

## End(Not run)
\end{ExampleCode}
\end{Examples}
\inputencoding{utf8}
\HeaderA{saga\_show\_tmpfiles}{List temporary files created by Rsagacmd}{saga.Rul.show.Rul.tmpfiles}
%
\begin{Description}
For convenience, functions in the Rsagacmd package create temporary files if
any outputs for a SAGA-GIS tool are not specified as arguments. Temporary
files in R are automatically removed at the end of each session. However,
when dealing with raster data, these temporary files potentially can consume
large amounts of disk space. These temporary files can be observed during a
session by using the saga\_show\_tmpfiles function, and can be removed using
the saga\_remove\_tmpfiles function.
\end{Description}
%
\begin{Usage}
\begin{verbatim}
saga_show_tmpfiles()
\end{verbatim}
\end{Usage}
%
\begin{Value}
returns the file names of the files in the temp directory that have
been generated by Rsagacmd. Note this list of files only includes the
primary file extension, i.e. '.shp' for a shapefile without the accessory
files (e.g. .prj, .shx etc.).
\end{Value}
%
\begin{Examples}
\begin{ExampleCode}
## Not run: 
# Show all temporary files generated by Rsagacmd
saga_remove_tmpfiles(h = 0)

## End(Not run)
\end{ExampleCode}
\end{Examples}
\inputencoding{utf8}
\HeaderA{saga\_version}{Return the installed version of SAGA-GIS}{saga.Rul.version}
%
\begin{Description}
Intended to be used internally by \code{\LinkA{saga\_env}{saga.Rul.env}}. Uses a system call
to saga\_cmd to output version of installed SAGA-GIS on the console
\end{Description}
%
\begin{Usage}
\begin{verbatim}
saga_version(saga_cmd)
\end{verbatim}
\end{Usage}
%
\begin{Arguments}
\begin{ldescription}
\item[\code{saga\_cmd}] The path of the saga\_cmd binary.
\end{ldescription}
\end{Arguments}
%
\begin{Value}
A numeric\_version with the version of SAGA-GIS found at the cmd path.
\end{Value}
\inputencoding{utf8}
\HeaderA{save\_object}{Generic methods to save R in-memory objects to file to SAGA-GIS to access}{save.Rul.object}
\keyword{internal}{save\_object}
%
\begin{Description}
Designed to be used internally by Rsagacmd for automatically pass data to
SAGA-GIS for geoprocessing.
\end{Description}
%
\begin{Usage}
\begin{verbatim}
save_object(x, ...)
\end{verbatim}
\end{Usage}
%
\begin{Arguments}
\begin{ldescription}
\item[\code{x}] An R object.

\item[\code{...}] Other parameters such as the temporary directory or the
vector/raster format used to write spatial datasets to file.
\end{ldescription}
\end{Arguments}
%
\begin{Value}
A character that specifies the file path to where the R object was
saved.
\end{Value}
\inputencoding{utf8}
\HeaderA{search\_saga}{Automatic search for the path to a SAGA-GIS installation}{search.Rul.saga}
%
\begin{Description}
Returns the path to the saga\_cmd executable.
\end{Description}
%
\begin{Usage}
\begin{verbatim}
search_saga()
\end{verbatim}
\end{Usage}
%
\begin{Details}
On Microsoft Windows, automatic searching will occur first in 'C:/Program
Files/SAGA-GIS'; 'C:/Program Files (x86)/SAGA-GIS'; 'C:/SAGA-GIS';
'C:/OSGeo4W'; and 'C:/OSGeo4W64'.

On Linux, saga\_cmd is usually included in PATH, if not an automatic search is
performed in the '/usr/' folder.

For macOS, since version 8.5, SAGA-GIS is available as an standalone macOS
app from \Rhref{https://sourceforge.net/projects/saga-gis/}{SourceForge}. The
'SAGA.app' package is searched first (assuming that it is installed in the
'/Applications/' folder). Other macOS locations that are searched include
'/usr/local/bin/' (for Homebrew installations) and within the QGIS application
(SAGA-GIS is bundled with the QGIS application on macOS by default).

If multiple versions of SAGA-GIS are installed on the system, the path to the
newest version is returned.
\end{Details}
%
\begin{Value}
The path to installed saga\_cmd binary.
\end{Value}
\inputencoding{utf8}
\HeaderA{search\_tools}{Search for a SAGA-GIS tool}{search.Rul.tools}
%
\begin{Description}
Search for a SAGA-GIS tool
\end{Description}
%
\begin{Usage}
\begin{verbatim}
search_tools(x, pattern)
\end{verbatim}
\end{Usage}
%
\begin{Arguments}
\begin{ldescription}
\item[\code{x}] saga object

\item[\code{pattern}] character, pattern of text to search for within the tool name
\end{ldescription}
\end{Arguments}
%
\begin{Value}
a tibble containing the libraries, names and parameters of the tools
that match the pattern of the search text and their host library
\end{Value}
%
\begin{Examples}
\begin{ExampleCode}
## Not run: 
# initialize Rsagacmd
saga <- saga_gis()

# search for a tool
search_tools(x = saga, pattern = "terrain")

## End(Not run)
\end{ExampleCode}
\end{Examples}
\inputencoding{utf8}
\HeaderA{show\_raster\_formats}{List the available raster formats that can be set as defaults for a `saga` object.}{show.Rul.raster.Rul.formats}
%
\begin{Description}
List the available raster formats that can be set as defaults for a `saga`
object.
\end{Description}
%
\begin{Usage}
\begin{verbatim}
show_raster_formats()
\end{verbatim}
\end{Usage}
%
\begin{Value}
tibble
\end{Value}
%
\begin{Examples}
\begin{ExampleCode}
show_raster_formats()
\end{ExampleCode}
\end{Examples}
\inputencoding{utf8}
\HeaderA{show\_vector\_formats}{List the available vector formats that can be set as defaults for a `saga` object.}{show.Rul.vector.Rul.formats}
%
\begin{Description}
List the available vector formats that can be set as defaults for a `saga`
object.
\end{Description}
%
\begin{Usage}
\begin{verbatim}
show_vector_formats()
\end{verbatim}
\end{Usage}
%
\begin{Value}
tibble
\end{Value}
%
\begin{Examples}
\begin{ExampleCode}
show_vector_formats()
\end{ExampleCode}
\end{Examples}
\inputencoding{utf8}
\HeaderA{summarize\_tool\_params}{Interval function used to summarize a `saga\_tool` into a tibble that describes the tools parameters and options}{summarize.Rul.tool.Rul.params}
\keyword{internal}{summarize\_tool\_params}
%
\begin{Description}
Interval function used to summarize a `saga\_tool` into a tibble that
describes the tools parameters and options
\end{Description}
%
\begin{Usage}
\begin{verbatim}
summarize_tool_params(tool_obj)
\end{verbatim}
\end{Usage}
%
\begin{Arguments}
\begin{ldescription}
\item[\code{tool\_obj}] a nested list which constitutes the internals of a saga\_tool
object
\end{ldescription}
\end{Arguments}
%
\begin{Value}
a tibble
\end{Value}
\inputencoding{utf8}
\HeaderA{tidy.saga}{Summarize the libraries that are available within a saga object and return these as a tibble.}{tidy.saga}
%
\begin{Description}
Summarize the libraries that are available within a saga object and
return these as a tibble.
\end{Description}
%
\begin{Usage}
\begin{verbatim}
## S3 method for class 'saga'
tidy(x, ...)
\end{verbatim}
\end{Usage}
%
\begin{Arguments}
\begin{ldescription}
\item[\code{x}] a `saga` object

\item[\code{...}] additional arguments. Currently unused.
\end{ldescription}
\end{Arguments}
%
\begin{Value}
a tibble that describes libraries, their descriptions and number of
tools that are available in SAGA-GIS.
\end{Value}
%
\begin{Examples}
\begin{ExampleCode}
## Not run: 
# Initialize a saga object
saga <- saga_gis()

# tidy the saga object's parameters into a tibble
tidy(saga)

## End(Not run)
\end{ExampleCode}
\end{Examples}
\inputencoding{utf8}
\HeaderA{tidy.saga\_library}{Summarize the tools that are available within a saga library and return these as a tibble.}{tidy.saga.Rul.library}
%
\begin{Description}
Summarize the tools that are available within a saga library and
return these as a tibble.
\end{Description}
%
\begin{Usage}
\begin{verbatim}
## S3 method for class 'saga_library'
tidy(x, ...)
\end{verbatim}
\end{Usage}
%
\begin{Arguments}
\begin{ldescription}
\item[\code{x}] a `saga\_library` object

\item[\code{...}] additional arguments. Currently unused.
\end{ldescription}
\end{Arguments}
%
\begin{Value}
a tibble that describes the tools and their descriptions within a
particular SAGA-GIS library.
\end{Value}
%
\begin{Examples}
\begin{ExampleCode}
## Not run: 
# Initialize a saga object
saga <- saga_gis()

# tidy the library's parameters into a tibble
tidy(saga$climate_tools)

## End(Not run)
\end{ExampleCode}
\end{Examples}
\inputencoding{utf8}
\HeaderA{tidy.saga\_tool}{Summarize the parameters that are available within a SAGA-GIS tool and return these as a tibble.}{tidy.saga.Rul.tool}
%
\begin{Description}
Summarize the parameters that are available within a SAGA-GIS tool and
return these as a tibble.
\end{Description}
%
\begin{Usage}
\begin{verbatim}
## S3 method for class 'saga_tool'
tidy(x, ...)
\end{verbatim}
\end{Usage}
%
\begin{Arguments}
\begin{ldescription}
\item[\code{x}] a `saga\_tool` object

\item[\code{...}] additional arguments. Currently unused.
\end{ldescription}
\end{Arguments}
%
\begin{Value}
a tibble that describes tools, identifiers used by the saga\_cmd
command line tool, the equivalent argument name used by Rsagacmd, and other
options and descriptions.
\end{Value}
%
\begin{Examples}
\begin{ExampleCode}
## Not run: 
# Initialize a saga object
saga <- saga_gis()

# tidy the tools parameters into a tibble
tidy(saga$ta_morphometry$slope_aspect_curvature)

## End(Not run)
\end{ExampleCode}
\end{Examples}
\inputencoding{utf8}
\HeaderA{tile\_geoprocessor}{Split a raster grid into tiles for tile-based processing}{tile.Rul.geoprocessor}
%
\begin{Description}
Split a raster grid into tiles. The tiles are saved as Rsagacmd
temporary files, and are loaded as a list of R objects for further
processing. This is a function to make the the SAGA-GIS
grid\_tools / tiling tool more convenient to use.
\end{Description}
%
\begin{Usage}
\begin{verbatim}
tile_geoprocessor(x, grid, nx, ny, overlap = 0, file_path = NULL)
\end{verbatim}
\end{Usage}
%
\begin{Arguments}
\begin{ldescription}
\item[\code{x}] A `saga` object.

\item[\code{grid}] A path to a GDAL-supported raster to apply tiling, or a
SpatRaster.

\item[\code{nx}] An integer with the number of x-pixels per tile.

\item[\code{ny}] An integer with the number of y-pixels per tile.

\item[\code{overlap}] An integer with the number of overlapping pixels.

\item[\code{file\_path}] An optional file file path to store the raster tiles.
\end{ldescription}
\end{Arguments}
%
\begin{Value}
A list of SpatRaster objects representing tiled data.
\end{Value}
%
\begin{Examples}
\begin{ExampleCode}
## Not run: 
# Initialize a saga object
saga <- saga_gis()

# Generate a random DEM
dem <- saga$grid_calculus$random_terrain(radius = 15, iterations = 500)

# Return tiled version of DEM
tiles <- tile_geoprocessor(x = saga, grid = dem, nx = 20, ny = 20)

## End(Not run)
\end{ExampleCode}
\end{Examples}
\inputencoding{utf8}
\HeaderA{update\_parameters\_file}{Updates a `parameters` object with file paths to the R data objects.}{update.Rul.parameters.Rul.file}
\keyword{internal}{update\_parameters\_file}
%
\begin{Description}
Updates a `parameters` object with file paths to the R data objects.
\end{Description}
%
\begin{Usage}
\begin{verbatim}
update_parameters_file(params, temp_path = NULL, raster_format, vector_format)
\end{verbatim}
\end{Usage}
%
\begin{Arguments}
\begin{ldescription}
\item[\code{params}] A `parameters` object.

\item[\code{temp\_path}] A character specifying the tempdir to use for storage
(optional).

\item[\code{raster\_format}] file extension for raster formats

\item[\code{vector\_format}] file extension for vector formats
\end{ldescription}
\end{Arguments}
%
\begin{Value}
A `parameters` object with updated `file` attributes that refers to
the on-disk file for saga\_cmd to access.
\end{Value}
\inputencoding{utf8}
\HeaderA{update\_parameters\_tempfiles}{Update a `parameters` object using temporary files for any unspecified output parameters}{update.Rul.parameters.Rul.tempfiles}
\keyword{internal}{update\_parameters\_tempfiles}
%
\begin{Description}
Update a `parameters` object using temporary files for any unspecified output
parameters
\end{Description}
%
\begin{Usage}
\begin{verbatim}
update_parameters_tempfiles(params, temp_path, raster_format, vector_format)
\end{verbatim}
\end{Usage}
%
\begin{Arguments}
\begin{ldescription}
\item[\code{params}] A `parameters` object.

\item[\code{temp\_path}] A character with the tempdir.

\item[\code{raster\_format}] A character specifying the raster format.

\item[\code{vector\_format}] A character specifying the vector format.
\end{ldescription}
\end{Arguments}
%
\begin{Value}
A `parameters` object.
\end{Value}
\inputencoding{utf8}
\HeaderA{update\_parameter\_file}{Updates a `parameter` object with file paths to the R data objects.}{update.Rul.parameter.Rul.file}
\keyword{internal}{update\_parameter\_file}
%
\begin{Description}
Updates a `parameter` object with file paths to the R data objects.
\end{Description}
%
\begin{Usage}
\begin{verbatim}
update_parameter_file(param, temp_path = NULL, raster_format, vector_format)
\end{verbatim}
\end{Usage}
%
\begin{Arguments}
\begin{ldescription}
\item[\code{param}] A `parameter` object.

\item[\code{temp\_path}] A character specifying the tempdir to use for storage
(optional).

\item[\code{raster\_format}] name of raster format in `supported\_raster\_formats`

\item[\code{vector\_format}] file extension for vector formats in
`supported\_vector\_formats`
\end{ldescription}
\end{Arguments}
%
\begin{Value}
A `parameter` object with an updated `file` attribute that refers to
the on-disk file for saga\_cmd to access.
\end{Value}
\printindex{}
\end{document}
